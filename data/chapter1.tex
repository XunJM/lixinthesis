\chapter{开始之前的准备}
\section{为什么要使用\LaTeX?}
\LaTeX 由来已久,但是很多人可能只闻其名,不见其身,简而言之就是用来排版的。也许很多人要说了,我用 Word 排版也非常方便啊,为什么要用\LaTeX 呢?任何东西都有它存在的意义,Word 也是\LaTeX 也是。Word 与\LaTeX 之间并没有孰优孰劣之分,很多时候做简单的文字编辑,Word 真的很方便。但是相对于一本书、一份长报告或者一篇长论文,\LaTeX 的优势就明显了起来。

比如导师喊你插一张图进去,这个时候用Word 的同学就面露难色,要知道插一张图进去以后,后面辛苦排的版就乱了啊!或者,导师让你删除一个章节或者修改一个章节,用Word 的同学修改了很久才将目录和正文部分修改好。再者,导师让你调整下文献先后顺序,用Word 的同学想死的心都有了,不仅要改文献,还要改前面引用的标志,天呐,怎么这么烦。

此时\LaTeX 的优势就体现了出来,\LaTeX 目录、章节、公式序号、引用等等等都是自动生成的,你要做的只是往里面添加内容就好了。最主要的,\textbf{用\LaTeX 编辑公式绝对比用Word 优美方便。}(作者在用Mathtype 的时候各种抱怨。。。)如果你愿意在前期花两天学习下\LaTeX, 那你在后期调整排版所花费的时间要比你使用Word 会少很多很多。

总之,你作为一个论文作者,写作的时候,内容才是你所要尽力关注的。不要被排版所困扰,这是\LaTeX 出现的目的。

\section{LXthesis 产生的契机}
作者在大一的时候由王立庆老师开始接触\LaTeX, 并对此产生了浓厚的兴趣,在学习了两年后,高瑞老师说:“既然你能力那么强,何不做一个模板出来?就当作是锻炼好了。”于是,这份模板就出现了。但是一个人的能力总是有限的,如果你在使用的过程中,有什么疑问、想法或者建议请发邮件至 {\sf xun.j@hotmail.com} 联系作者。帮助改善 LXthesis, 也算是为后辈们尽一份绵薄之力。

\section{搭建\LaTeX 环境}
暂未测试,后续添加

\section{编译方式}
本模板使用 XeLaTeX 编译,具体如下:

\section{中英文格式}
要记住,在中文中插入英文,记得前后空一格。如果该句由英文单词结束,直接半角逗号或者半角句号再空格,你可以打开这个模板的源文件参考。

中文首行要空格,所以在源文件中,每段中间要空行。强制首行缩进命令为$\backslash$indent, 但是必须在上一行结尾$\backslash\backslash$或者 $\backslash$newline 强制换行,所以不推荐此用法。

\section{多读书,勤用搜索引擎}
阅读一些基础的\LaTeX 教程,比如本模板附带的\emph{一份不太简短的\LaTeXe 介绍}。遇到不懂的问题先上百度,Google 搜索。

注意,本文也是用 LXthesis 生成的,所以整个压缩包中含有该手册的所有内容,你只需依葫芦修改即可。

那么接下来正式进入使用手册,告诉你怎么使用LXthesis. 