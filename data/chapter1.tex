\chapter{开始之前的准备}
\section{为什么要使用\LaTeX?}
\LaTeX 由来已久,但是很多人可能只闻其名,不见其身,简而言之就是用来排版的。也许很多人要说了,我用 Word 排版也非常方便啊,为什么要用\LaTeX 呢?任何东西都有它存在的意义,Word 也是\LaTeX 也是。Word 与\LaTeX 之间并没有孰优孰劣之分,很多时候做简单的文字编辑,Word 真的很方便。但是相对于一本书、一份长报告或者一篇长论文,\LaTeX 的优势就明显了起来。

比如导师喊你插一张图进去,这个时候用Word 的同学就面露难色,要知道插一张图进去以后,后面辛苦排的版就乱了啊!或者,导师让你删除一个章节或者修改一个章节,用Word 的同学修改了很久才将目录和正文部分修改好。再者,导师让你调整下文献先后顺序,用Word 的同学想死的心都有了,不仅要改文献,还要改前面引用的标志,天呐,怎么这么烦。

此时\LaTeX 的优势就体现了出来,\LaTeX 目录、章节、公式序号、引用等等等都是自动生成的,你要做的只是往里面添加内容就好了。最主要的,\textbf{用\LaTeX 编辑公式绝对比用Word 优美方便。}(作者在用Mathtype 的时候各种抱怨。。。)如果你愿意在前期花两天学习下\LaTeX, 那你在后期调整排版所花费的时间要比你使用Word 会少很多很多。

总之,你作为一个论文作者,写作的时候,内容才是你所要尽力关注的。不要被排版所困扰,这是\LaTeX 出现的目的。

\section{lixinthesis 产生的契机}
作者在大一的时候由数学经济的老师开始接触\LaTeX, 并对此产生了浓厚的兴趣,在学习了两年后,我的辅导员说:“既然你能力那么强,何不做一个模板出来?就当作是锻炼好了。”于是,这份模板就出现了。但是一个人的能力总是有限的,如果你在使用的过程中,有什么疑问、想法或者建议请发邮件至 {\sf xun.j@hotmail.com} 联系作者。帮助改善 lixinthesis, 也算是为后辈们尽一份绵薄之力。

\section{搭建\LaTeX 环境}
入门\LaTeX 的一个难点,就是搭建一个\LaTeX 环境。这里只讲如何在 Windows 下搭建一个易用的环境。

一个完整的\LaTeX 环境需要有以下3个东西:前端编辑器、\LaTeX 发行版和 PDF 浏览器。最快的且最方便的搭建办法就是下载 CTEX 套装,百度搜索下自行安装即可,但是版本老旧,而且无法编译本模板(会更新一个 CTEX 可用的模板),所以这里不推荐使用。

笔者在这里推荐以下组合:TexStudio+TexLive2016+SumatraPDF. TexStudio 是一款相当不错的编辑器,具体使用方法请百度。 TexLive 是一个相当不错的 \LaTeX 发行版,你可以百度下载(注意,下载 ISO 镜像文件安装),也可以用本模板附带的种子文件下载,安装时请不要安装 TexWorks 前端(这也是一个编辑器,只是没有 TexStudio 好用,看个人喜好吧)。最后就是一款小巧全能的 PDF 浏览器了——SumatraPDF, 当然用别的也可以,主要是这个支持动态预览。

\section{编译方式}
本模板使用 XeLaTeX 编译,具体如下:

\section{中英文格式}
要记住,在中文中插入英文,记得前后空一格。如果该句由英文单词结束,直接半角逗号或者半角句号再空格,你可以打开这个模板的源文件参考。

中文首行要空格,所以在源文件中,每段中间要空行。强制首行缩进命令为$\backslash$indent, 但是必须在上一行结尾$\backslash\backslash$或者 $\backslash$newline 强制换行,所以不推荐此用法。

\section{多读书,勤用搜索引擎}
阅读一些基础的\LaTeX 教程,比如本模板附带的\emph{一份不太简短的\LaTeXe 介绍}。遇到不懂的问题先上百度,Google 搜索。

注意,本文也是用 lixinthesis 生成的,所以整个压缩包中含有该手册的所有内容,你只需依葫芦修改即可。

那么接下来正式进入使用手册,告诉你怎么使用 lixinthesis. 