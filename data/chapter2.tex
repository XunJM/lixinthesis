\chapter{正式开始}
\section{本模板的结构}
\subsection{封面}
由命令$\backslash$maketitle 生成,具体内容在 data 目录下的 cover 文档中修改。

一并由该目录生成的还有 声明及论文使用的授权 页。

\subsection{目录}
由命令$\backslash$tableofcontents 生成,在根目录 main 文档中,不建议做修改。

\subsection{摘要}
由命令$\backslash$cnabstract 和$\backslash$enabstract 生成,具体内容在 data 目录下的 abstract 文档中修改。

\subsection{正文}
本模板推荐将每个章节都写成一个 tex 文档且自定义命名(不用像本模板这样以 chapter1, chapter2 命名)并保存在 data 下,方便调整顺序以及用$\backslash$input 命令导入主文档。

本模板的章节结构层次为$\backslash$chapter 章,$\backslash$section 节,$\backslash$subsection 目,$\backslash$subsubsection 条。

\subsection{参考文献}
由于国内杂乱的文献数据库,请自行手动编辑参考文献,具体要求略。

\subsection{致谢}
具体内容请在 data 目录下 thank 文档中更改。

\subsection{附录}
由命令$\backslash$appendix 生成,具体内容在 data 目录下的 appendix 文档中修改。

\section{本模板自定义的命令}
本模板自定义了一些命令以方便使用。

\subsection{字体和字号}
如\cref{zt}所示,
    
\begin{longtable}{lr}
	\caption{命令对应的字体\label{zt}}\\
	\hline
	 命令 & 效果\\
	\hline
	\endfirsthead %定义第一页表头
	\hline
	 命令 & 效果\\
	\hline
	\endhead %定义每一页表头
	\hline   %定义分页处横线
	\endfoot %定义分页处横线
	\{$\backslash$rm 宋体 Times New Roman\} & {\rm 宋体 Times New Roman}\\
	\{$\backslash$sf 黑体 Arial\} & {\sf 黑体 Arial}\\
	\{$\backslash$song 宋体\} & {\song 宋体}\\
	\{$\backslash$hei 黑体\} & {\hei 黑体}\\
	\{$\backslash$kai 楷体\} & {\kai 楷体}\\
	\{$\backslash$li 隶书\} & {\li 隶书}\\
	\{$\backslash$fs 仿宋\} & {\fs 仿宋}\\ 
	\{$\backslash$hwxw 华文新魏\} &{\hwxw 华文新魏}\\
	\{$\backslash$qihao 七号\} & {\qihao 七号}\\
	\{$\backslash$liuhao 六号\} & {\liuhao 六号}\\
	\{$\backslash$xiaowuhao 小五号\} & {\xiaowuhao 小五号}\\
	\{$\backslash$wuhao 五号\} & {\wuhao 五号}\\
	\{$\backslash$xiaosihao 小四号\} & {\xiaosihao 小四号}\\
	\{$\backslash$sihao 四号\} & {\sihao 四号}\\
	\{$\backslash$xiaosanhao 小三号\} & {\xiaosanhao 小三号}\\
	\{$\backslash$sanhao 三号\} & {\sanhao 三号}\\
	\{$\backslash$xiaoerhao 小二号\} & {\xiaoerhao 小二号}\\
	\{$\backslash$erhao 二号\} & {\erhao 二号}\\
	\{$\backslash$xiaoyihao 小一号\} & {\xiaoyihao 小一号}\\
	\{$\backslash$yihao 一号\} & {\yihao 一号}\\
	\{$\backslash$xiaochuhao 小初号\} & {\xiaochuhao 小初号}\\
	\{$\backslash$chuhao 初号\} & {\chuhao 初号} \\
	\hline
\end{longtable}

如果你有兴趣的话可以研究下字体,比如等宽字体,衬线字体什么的,很有意思的东西。

\subsection{注释和引用}
使用命令 $\backslash$footnote\{解释内容\} 添加注释,本模板一页能标记9个注释且每页刷新序号\footnote{这里是解释内容}。

使用命令 $\backslash$label\{标记\}\footnote{用自己定义的好记得名字或者缩写,尽量别用数字,会乱的,可以看本模板的源代码学习}标记图、表和公式的名字,再使用命令 $\backslash$cref\{标记\} 来引用。这个做法的好处是,全自动生成图、表和公式的序号,无需人为添加和修改。

注意,$\backslash$cref命令对eqnarray环境的支持不好\footnote{详见cleveref宏包说明文档},多行公式请使用align环境而避免使用eqnarray环境\footnote{详见 Lars Madsen 在2006年发表的 \emph{Avoid eqnarray!} }。

如\cref{eq1}和\cref{eq2}:
\begin{align}
1=0\label{eq1}\\
0=1\label{eq2}
\end{align}

使用命令 $\backslash$cite\{文献标记\} 来引用文献,如\cite{yfbt}.

\subsection{插入图片和表格}
请打开本手册的源代码查看。
插入图片如\cref{fig.ex1}。
\begin{figure}[htbp]
	\centering\includegraphics[width=0.2\textwidth]{logo1.jpg}
	\caption{插图示例}\label{fig.ex1}
\end{figure}

插入子图,效果如\cref{fig.ex2}所示。
\begin{figure}[htbp]
	\centering
	\subfigure[校logo]{
		\label{Fig.sub.1}
		\includegraphics[width=0.2\textwidth]{logo1.jpg}}
	\subfigure[校名]{
		\label{Fig.sub.2}
		\includegraphics[width=0.4\textwidth]{logo2.jpg}}
	\caption{Logo样张}
	\label{fig.ex2}
\end{figure}

示例表格如\cref{bg}:
\begin{longtable}{|c|c|c|c|c|}	
	\hline
	项目1 & 项目2 & 项目3 & 项目4 & 项目5 \\
	\hline
	\endhead
    元素1 & 元素2 & 元素3 & 元素4 & 元素5 \\
    \hline
    元素6 & 元素7 & 元素8 & 元素9 & 元素10 \\
    \hline
    元素11 & 元素12 & 元素13 & 元素14 & 元素15 \\
    \hline
	\caption{示例表格\label{bg}}
\end{longtable}

表格的制作也是一本学问,其他样式请善用Google, 百度学习。

\section{注意}
\LaTeX 在编译过程中是区分大小写的!



