\phantomsection
\addcontentsline{toc}{chapter}{引言}
\chapter*{引言:为什么要使用\LaTeX?}
\LaTeX 由来已久,但是很多人可能只闻其名,不见其身,简而言之就是用来排版的。也许很多人要说了,我用 Word 排版也非常方便啊,为什么要用\LaTeX 呢?任何东西都有它存在的意义,Word 也是\LaTeX 也是。Word 与\LaTeX 之间并没有孰优孰劣之分,很多时候做简单的文字编辑,Word 真的很方便。但是相对于一本书、一份长报告或者一篇长论文,\LaTeX 的优势就明显了起来。

比如导师喊你插一张图进去,这个时候用Word 的同学就面露难色,要知道插一张图进去以后,后面辛苦排的版就乱了啊!或者,导师让你删除一个章节或者修改一个章节,用Word 的同学修改了很久才将目录和正文部分修改好。再者,导师让你调整下文献先后顺序,用Word 的同学想死的心都有了,不仅要改文献,还要改前面引用的标志,天呐,怎么这么烦。

此时\LaTeX 的优势就体现了出来,\LaTeX 目录、章节、公式序号、引用等等等都是自动生成的,你要做的只是往里面添加内容就好了。最主要的,\textbf{用\LaTeX 编辑公式绝对比用Word 优美方便。}(作者在用Mathtype 的时候各种抱怨。。。)如果你愿意在前期花两天学习下\LaTeX, 那你在后期调整排版所花费的时间要比你使用Word 会少很多很多。

总之,你作为一个论文作者,写作的时候,内容才是你所要尽力关注的。不要被排版所困扰,这是\LaTeX 出现的目的。


